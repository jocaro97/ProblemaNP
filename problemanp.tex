\documentclass[11pt]{article}

\usepackage[table,xcdraw]{xcolor}
%Gummi|065|=)
\title{\textbf{Conjunto diverso de clientes}}
\author{Johanna Capote Robayna\\
		Guillermo Galindo Ortuño}
\date{}
\begin{document}

\maketitle

\section*{Problema}
Un comerciante mantiene una matriz de compras de sus clientes. Hay una fila por cada cliente y una columna por cada producto y en cada entrada $(i,j)$ regustra el número de unidades que el cliente $i$ ha comprado del producto $j$. Un subconjunto $S$ de cliente se llama diverso si nunca dos clientes de $S$ distintos han comprado alguna vez el mismo producto. El problema consiste en dado $K$, determinar si existe un conjunto $S$ de clientes de tamaño $K$ que sea diverso. \\
\begin{center}
\textbf{3SET} $\propto$ \textbf{Conjunto diverso de clientes} \\
$X = \{ a_1, \dots, a_{3n} \} \rightarrow  Productos = \{p_1, \dots , p_{3n} \}  $ \\
$ n \rightarrow K$ \\
$C = \Big\{ \{ b_{11},b_{12}, b_{13}\} , \dots , \{b_{m1}, b_{m2}, b_{m3} \} \Big\} \rightarrow Clientes = \{ c_1, \dots , c_m \} \}$
\end{center}

Formamos una tabla, ponemos en las columnas los elementos $a_i$ y en las filas los $b_j$. 
Por cada $b_j$ marcamos con un 1 cada elementos de X que forman parte del conjunto. Si $b_1 = \{ a_1, a_2, a_4 \} $ entonces la tabla quedaria de la siguiente forma:
\begin{table}[h]
\centering
\begin{tabular}{c|c|c|c|c|}
\cline{2-5}
                            & $a_1$ & $a_2$ & $a_3$ & $a_4$ \\ \hline
\multicolumn{1}{|c|}{$b_1$} & 1     &       & 1     & 1     \\ \hline
\end{tabular}
\end{table}
\\
Por lo tanto en el problema del conjunto diverso de clientes, cada cliente siempre tiene asociado 3 productos.
\\
\textbf{3SET tiene solución} $\Rightarrow$ \textbf{Conjuntos diverso de clientes tiene solución} \\
Si el problema de 3SET tiene solución significa que existe $C' \subseteq C$ de tamaño $n$ tal que $\cup b_i = X$ con $b_i \cap b_j = \emptyset $ si $b_i, b_j \in C'$ con $i\neq j$. Por lo tanto existe un conjunto de $K$ Clientes tal que sus Productos no coinciden, es decir que si un cliente ha comprado algún producto $p$ ningún otro cliente del conjunto lo ha comprado, lo que implica que este subconjunto $C'$ es diverso, dejando resuelto el problema del Conjunto diverso de clientes.
\\
\textbf{Conjuntos diverso de clientes tiene solución} $\Rightarrow$ \textbf{3SET tiene solución} \\ 
Si el problema del conjunto diverso de clientes tiene solución significa que existe un subconjunto de clientes de tamaño $K$ diverso, esto quiere decir existe $K$ elementos de $C$  tal que $b_i \cap b_j = \emptyset$ si $i \neq j$. Solo nos queda comprobar que $\cup b_i = X$, esto es cierto puesto que $K = n$ por lo tanto hemos elegido $n$ elementos de $C$ de 3 elementos, consiguiendo los $3n$ elementos de $X$.\\
\section*{Ejemplo}
\textbf{3SET} \\
$X = \{a_1, a_2, a_3, a_4, a_5, a_6, a_7, a_8, a_9\}$ \\
$C = \Big\{ \{a_1, a_2, a_3 \}, \{a_2, a_3, a_4 \}, \{ a_1, a_5, a_6 \}, \{a_7,a_8,a_9\}, \{a_3,a_6,a_8\} \Big\}$

\begin{table}[h]
\centering
\begin{tabular}{c|c|c|c|c|c|c|c|c|c|}
\cline{2-10}
                                                    & $a_1$ & $a_2$ & $a_3$ & $a_4$ & $a_5$ & $a_6$ & $a_7$ & $a_8$ & $a_9$ \\ \hline
\multicolumn{1}{|c|}{$b_1$}                         & 1     &       & 1     & 1     &       &       &       &       &       \\ \hline
\rowcolor[HTML]{9AFF99} 
\multicolumn{1}{|l|}{\cellcolor[HTML]{9AFF99}$b_2$} &       & 1     & 1     & 1     &       &       &       &       &       \\ \hline
\rowcolor[HTML]{9AFF99} 
\multicolumn{1}{|l|}{\cellcolor[HTML]{9AFF99}$b_3$} & 1     &       &       &       & 1     & 1     &       &       &       \\ \hline
\rowcolor[HTML]{9AFF99} 
\multicolumn{1}{|c|}{\cellcolor[HTML]{9AFF99}$b_4$} &       &       &       &       &       &       & 1     & 1     & 1     \\ \hline
\multicolumn{1}{|c|}{$b_5$}                         &       &       & 1     &       &       & 1     &       & 1     &       \\ \hline
\end{tabular}
\end{table}


\textbf{Solución: } $C' = \{ b_2, b_3, b_4 \}$ \\ \\
\textbf{Conjuntos diverso de clientes} \\
$Productos = \{a_1, a_2, a_3, a_4, a_5, a_6, a_7, a_8, a_9\}$ \\
$Clientes = \{b_1, b_2, b_3, b_4, b_5 \}$ \\
\textbf{Solución: } $C' = \{ b_2, b_3, b_4 \}$ 


\end{document}
